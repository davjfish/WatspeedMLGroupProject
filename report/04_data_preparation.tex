\section{Data Preparation}

\subsection{What was your data source?}


The data used for this analysis originate from archives of images taken of otoliths from two Atlantic Canadian species of fish: Atlantic Herring and American Plaice.
While these archives are unpublished, some of the details about how the otoliths were collected, preserved and imaged are available on the Government of Canada's Open Data Portal ~\cite{ogp_plaice}.
The fish specimens themselves were collected from various Canadian government stock assessment surveys (e.g., ~\cite{groundfish} and ~\cite{ns_strait}) as well as from commercial port sampling programs.

\subsection{Data quality and procurement?}

The quality of the data was very high.
All the specimens had unique universal identifiers (UUIDs) which corresponded to the image file names.
None of the images used were orphaned, i.e., without corresponding metadata.

The dataset contained ages that were evaluated by human experts and which are actively used in annual population models for the two species in question.
Furthermore, since each fish specimen has two otoliths, our target dataset was effectively doubled since for every fish whose age is known, there were two corresponding images.

The data was procured by a team member who has a role at Fisheries and Oceans Canada in maintaining and archiving the dataset.

\subsection{What tools or code did you need to use to prepare it for analysis?}

Here is a list of the python tools used for processing, data preparation and model construction:

\begin{itemize}
    \item Jupyter Notebooks: for the development and presentations of our analysis
    \item OpenCV: for image processing and manipulation.
    \item CNN (TensorFlow / Keras): for deep learning and automatic feature extraction from images.
    \item XGBoost: for gradient-boosting regression modeling.
    \item Scikit-learn: for machine learning models, data preprocessing, and evaluation
    \item Pandas: for handling and manipulating tabular data
    \item Matplotlib / Seaborn: for visualizing data and model performance.
\end{itemize}

\subsubsection{Fish Specimen Data}

Information about the fish specimen was stored in a CSV file which contained metadata for each otolith, including the UUID, age, length, weight, and other categorical features such as gender and species.
The categorical variables were already converted into dummy variables.
For example, at the time of collection, a specimen's sex is identified as either `male`, `female` or `unknown`.
In the dataset, these categories were represented by three mutually exclusive columns: `is\_male`, `is\_female` and `is\_unknown`.
The species were represented by two mutually exclusive columns: `is\_plaice` and `is\_herring`.
As noted above, the metadata is linked to the corresponding otolith image based on the UUID\@.
The CSV file was simply read into a pandas dataframe.
In Table \ref{tab:features} a detailed explanation of each column in the dataset is provided.

\subsubsection{Otolith Images}

The dataset contained a total of 15,089 images of otoliths originating from the 7828 specimens.
Therefore, on average, each specimen had 1.9 corresponding otoliths.
The otolith image was identified by a UUID prefix in its filename.
The original resolution of the images were 1000 x 1000 pixels and were in RGB.
To make the image processing more manageable for this project, they were resized to 64 x 64 pixels and to a greyscale image.
The image was subsequently flattened from a 1D array to a 2D array.
Similarly, for the purposes of shortening the processing time of the scripts, only one randomly selected image from each specimen was used in the analysis.
Finally, the images were read into a 7,828 x 4097 pandas dataframe, where the rows were the image instances and the columns were the pixels plus one extra one for the UUID which was parsed from the filename.
This was then merged with the metadata dataframe referenced above using a left join on the `uuid` column.
After the images and metadata were merged, the numerical features of the resulting matrix were normalized.
Specifically, we used the StandardScaler to normalize the numerical features (e.g, length, weight and pixel columns) meaning those columns will have zero mean and unit variance.
This step is crucial for machine learning models to perform optimally.
Finally, the dataset was split into training and testing portions using the `train\_test\_split` according to a ratio of 4:1, respectively.
This resulted in 6262 instances for model training and the remaining 1566 instances for model testing.

\subsection{What challenges did you face?}

One of the main significant challenges encountered during this analysis was the large size of the dataset.
With over 15,000 images totaling nearly 4 GB, managing and processing the data was difficult.
The sheer volume of data made it difficult to disseminate and significantly slowed down the model training process, making it clear that resizing the images was necessary.
As a result, we had to resize the images to very small sizes to achieve reasonable computation times.
However, smaller images likely led to the loss of important details, which could potentially decrease accuracy.
Another challenge involved managing multiple images per UUID.
Since each specimen typically had more than one image, it was necessary to determine the most effective method for utilizing them.

\\~\\
Three approaches were explored:
\\~\\


\begin{enumerate}
    \item \textbf{Random Selection:} Selecting a single image at random per UUID.
    \item \textbf{Image Concatenation:} Merging multiple images into a single, larger input.
    \item \textbf{Image Averaging:} Averaging the images to generate a single representative image.
\end{enumerate}

Each approach presented its own limitations.
Random selection could overlook valuable information, concatenation introduced additional complexity, and averaging risked the loss of critical features.
Balancing accuracy with efficiency in choosing the optimal method proved to be a time-consuming and challenging task.



\begin{longtable}{| p{.15\textwidth} | p{.15\textwidth} | p{.30\textwidth} | p{.30\textwidth} |}
    \caption{Detailed explanation of the column in the specimen metadata dataset}
    \label{tab:features} \\
    \hline
    Name &
    Type &
    Description &
    Purpose \\
    \hline
    \hline
    uuid &
    String (or unique identifier) & A unique identifier for each data record.
    This is useful for tracking and referencing individual fish samples in the dataset. &
    Helps ensure each record is uniquely identifiable.
    It would not be used directly in the model for predictions, but it's essential for managing the dataset.  \\
    \hline
    fish\_id &
    String &
    A unique identifier for each individual fish. &
    Individual fish records.
    It might be useful for aggregation or analysis purposes but would generally not be used in the model itself. \\
    \hline
    age &
    Integer & The known age of the fish, which will be used as the target variable for the model. &
    The primary target variable to predict based on the otolith image features.
    The model will learn to predict this based on otolith image analysis and possibly other features. \\
    \hline
    length &
    Float &
    The length of the fish, which is often correlated with age and could be useful as a feature for the model. &
    Features like length and weight can help the model better estimate the age since they provide additional biological context for each fish. \\
    \hline
    weight &
    Float &
    The weight of the fish, which, like length, may correlate with age. &
    Another potentially important feature for age estimation, as larger or heavier fish may be older. \\
    \hline
    month &
    Integer &
    the specimen was collected (from the wild) normalized to 1 &
    A potential feature for the model to understand seasonal variations in growth or age estimation. \\
    \hline
    is\_male &
    Boolean (0 or 1) &
    A binary value indicating whether the fish is male (1) or not (0). &
    Gender could influence growth rates and age estimation, so this feature might improve model accuracy. \\
    \hline
    is\_female &
    Boolean (0 or 1) &
    A binary value indicating whether the fish is female (1) or not (0). &
    Like is\_male, gender could affect age estimation, especially in species where males and females have different growth patterns. \\
    \hline
    is\_unknown &
    Boolean (0 or 1) &
    A binary value indicating unknown (1) or known (0). &
    Gender information is unavailable for these instances. \\
    \hline
    is\_plaice &
    Boolean (0 or 1) &
    A binary value indicating whether the fish is an American plaice (1) or not (0). &
    Species-specific characteristics could influence the model, as growth patterns may differ between species.
    This feature will help the model differentiate between the two species. \\
    \hline
    is\_herring &
    Boolean (0 or 1) &
    A binary value indicating whether the fish is an Atlantic herring (1) or not (0). &
    Similar to is\_plaice, this feature will help the model differentiate between the two species (American plaice and Atlantic herring), as their growth rates, lifespan, and annuli in otoliths may differ. \\
    \hline


\end{longtable}

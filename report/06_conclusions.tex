\section{Conclusions}

\subsection{Was the model useful?}

In a laboratory setting, agers are tasked with assessing the ages of biological specimens with nothing but the image of that specimen's otolith.
This is by design.
The reason is we want to make sure that their assessments are unbiased by biological information about the fish (e.g., length, weight, etc.).
Since there is a direct 1:1 relationship between annuli and fish age, this is really the best and truest form of age determination.
It is not surprising that the models that contained the metadata performed much better than those that did not.
There is a strong and well-documented relationship between age and fish size.
That being said, this relation will vary across time and space.
There are many factors that can influence the size-at-age of a fish population, chief among them are habitat quality and predation.
Accordingly, a truly useful model should play by the same rules as human agers, i.e., be based solely on the image of the otolith.
In other words, these models need to be able to decode the age from an otolith image, just as a human does.
While we took steps in the right direction with our project, we did not produce a model that would have any value in a production context.

\subsection{What did you learn about your data set?}

The lesson that stood out the most is that working with images can be quite challenging both from a model complexity point of view but also from a logistic point of view.
The datasets get large very fast and require large computing capacity.
There were a lot of questions and experimental approaches that were simply not explored due to practical limitations.
We also noted that the convolutional neutral networks (CNNs) displayed a much higher capacity for capturing the complexity of the data when compared to the other types of models tested.
However, more work is needed for determining how CNN models can be fine-tuned to determine age from otolith images.
It is only at this point where the deployment of these models in a production setting can be realistically considered.
Furthermore, an in-depth exploration of the strengths and weaknesses of these models will need to be well understood.
For example, under which circumstances are these models likely to succeed or fail?
Are the ages of certain species more difficult to decode than others?

\subsection{What would you do next to improve your model?}

For future analyses, we plan to explore tools designed for efficiently handling large datasets through distributed computing.
Instead of relying on a local setup like Jupyter Notebooks, we could leverage platforms that support scalable computation across multiple servers.
This would enable parallel processing, significantly improving the speed of data analysis and model training.
As a result, we could minimize the need to downsize images, preserving model accuracy without excessive computational time.
Additionally, this would provide more flexibility to experiment with different models and algorithms, leading to better and more optimized solutions.
Furthermore, implementing ensemble methods could further enhance accuracy, and incorporating cross-validation would help ensure better generalization and consistency across different data subsets, ultimately improving the model's ability to predict fish age accurately.


